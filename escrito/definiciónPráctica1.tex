%
%  untitled
%
%  Created by Sergio on 2009-03-13.
%  Copyright (c) 2009 __MyCompanyName__. All rights reserved.
%
\documentclass[]{article}

% Use utf-8 encoding for foreign characters
\usepackage[utf8]{inputenc}

% Setup for fullpage use
\usepackage{fullpage}

% Uncomment some of the following if you use the features
%
% Running Headers and footers
%\usepackage{fancyhdr}

% Multipart figures
%\usepackage{subfigure}

% More symbols
%\usepackage{amsmath}
%\usepackage{amssymb}
%\usepackage{latexsym}

% Surround parts of graphics with box
\usepackage{boxedminipage}

% Package for including code in the document
\usepackage{listings}

% If you want to generate a toc for each chapter (use with book)
\usepackage{minitoc}

% This is now the recommended way for checking for PDFLaTeX:
\usepackage{ifpdf}

%\newif\ifpdf
%\ifx\pdfoutput\undefined
%\pdffalse % we are not running PDFLaTeX
%\else
%\pdfoutput=1 % we are running PDFLaTeX
%\pdftrue
%\fi

\ifpdf
\usepackage[pdftex]{graphicx}
\else
\usepackage{graphicx}
\fi
\title{Definici\'on Pr\'actica 1}
\author{  }

\date{2009-03}

\begin{document}

\ifpdf
\DeclareGraphicsExtensions{.pdf, .jpg, .tif} 
\else
\DeclareGraphicsExtensions{.eps, .jpg}
\fi

\maketitle


\section{Descripci\'on General}
Simulador de modelos at\'omicos (Thomson, Rutherford, Sommerfeld, Schr\"odinger y Bohr).
\section{Descripci\'on Funcionalidad}
- La aplicaci\'on deber\'a permitir al usuario escoger alguno de los modelos at\'omicos.\\
Identificaci\'on: RFUN-01\\
Tipo: Funcional.\\
Descripci\'on: La aplicaci\'on permite escoger entre los distitos modelos at\'omicos (Thomson, Rutherford, Sommerfeld, Schr\"odinger y Bohr) puediendo ver como es su interacci\'on.\\
Justificaci\'on:  Poder comparar los modelos\\
El Dependencia:    Ninguna.\\
Conflictos: Ninguno.\\
Fuente: Sebasti\'an Arcila.\\
Prioridad: 2\\
Satisfacci\'on del Cliente: 5\\
Insatisfacci\'on del Cliente: 2\\
Hist\'orico: Sebasti\'an Arcila 2009-03-13\\\\
- La aplicaci\'on permite interactuar con las constantes y n\'umero de elementos en cada modelo.\\
Identificaci\'on: RFUN-02.\\
Tipo: Funcional.\\
Descripci\'on: La aplicaci\'on implementar\'a la posibilidad que el usuario mediante un evento pueda agregar unas nuevas condiciones a cada uno de los modelos.\\
Justificaci\'on:  Para entender mejor cada modelo y ver como se agustan algunas aproximaciones entre ellos.\\
Dependencia: Ninguna.\\
Conflictos: Ninguno.\\
Fuente: Sebasti\'an Arcila.\\
Prioridad: 4\\
Satisfacci\'on del Cliente: 5\\
Insatisfacci\'on del Cliente: 4\\
Hist\'orico: Sergio Botero 2009-03-13\\

\section{IHerramientas}
\begin{itemize}
\item  OpenGL sobre C++ (set completo de OpenGL)
\item Build Essentials
\item Makefile
\item Emacs
\end{itemize}
\section{Integrantes}
\begin{itemize}
\item Sebasti\'an Arcila V. 200710029010
\item Sergio Botero U. 200710001010
\end{itemize}
\section{Cronograma Semanal}


\begin{tabular}{cccc}
\hline
Semana & Tareas &  Encargados & \\
\hline
1 (16-20 Marzo) & Estudio de Modelos at\'omicos y revisar opciones de implementaci\'on & Arcila - Botero & \\
\hline
2 (16-20 Marzo) & Implementaci\'on de la aplicaci\'on & Arcila - Botero & \\
\hline
3 (30-3 Marzo) & Entrega & Arcila - Botero & \\
\hline
\end{tabular}


\bibliographystyle{plain}
\bibliography{}
\end{document}
